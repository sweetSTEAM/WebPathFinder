\section{Введение}
Поиск пути является одной из важнейших задач в теории графов. Решение данной задачи имеет широкое практическое применение в современных технологиях. В любой сфере разработки, в которой рабочее пространство можно представить в виде графа, реализация поиска пути является одним из основных аспектов.

Представление сети дорог ориентированным графом с положительными весами и возможностью изменения веса ребёр позволяет разработчикам картографических сервисов решить данную задачу с учетом расположения физических объектов, длины дорог, их типа, проходимости, наличия пробок. Алгоритмы маршрутизации, с помощью которых информация находит свой путь от одного устройства к другому, также основываются на теории графов и задаче поиска пути. Создание искусственного интеллекта во многих играх не обходится без поиска пути между объектами на игровой карте.

Первые алгоритмы, позволяющие оптимально решить данную задачу, начали появляться в конце 50-ых — начале 60-ых годов XX века. Один из самых известных, алгоритм Дейкстры, был изобретён в 1959 и стал основой для многих последующих алгоритмов поиска пути. В 1968 году появилось его улучшение: алгоритм A*, который также приобрёл широкую популярность за счёт более быстрой работы в условиях определённости конечной точки.

Необходимость ускорения и оптимизации работы алгоритмов поиска пути вызвало появление огромного количество их реализаций с различными структурами хранения вершин.
