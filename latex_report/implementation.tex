\subsection{Реализация}
\subsubsection{Список используемых технологий}
\begin{enumerate}
\item JavaScript — интерпретируемый браузером язык программирования.
\item HTML5, CSS — языки разметки для создания интерфейса.
\item Bootstrap, jQuery — подключённые библиотеки.
\item LaTeX — система компьютерной верстки, значительно облегчающая создание технической литературы. Используется для создания отчёта.
\item VCS Git — система контроля версий.
\end{enumerate}
\subsubsection{Язык программирования и библиотеки}
Для реализации поставленных задач был выбран язык JavaScript. Его программой-интерпретатором является браузер (ссылка). Язык прост в освоении, имеет похожий на язык C синтаксис, элементы объектно-ориентированного программирования, динамическую типизацию переменных, позволяет работать с обработчиками событий(сслыка).

JavaScript поддерживается всеми современными браузерами, что обеспечивает реализованной программе кроссплатформенность и возможность запуска даже на мобильных устройствах.

По соображениям безопасности, на язык наложены некоторые ограничения. Он не имеет прямого доступа к операционной системе. Например, возможность чтения и записи файлов сильно ограничено.

Язык имеет полную интеграцию с языком разметки HTML и CSS. Первый позволяет создавать прототип интерфейса, его скелет, а второй определяет внешний вид объектов интерфейса.

Вместе тройка данных языков предоставляет невероятные возможности для построения интерфейса и визуализации, что хорошо подходит для осуществления поставленных задач.

Для упрощения работы подключены библиотеки jQuery и Bootstrap. jQuery используется для быстрого доступа к содержимому HTML и необходим для работы Bootstrap. Bootstrap предоставляет готовые решения для элементов интерфейса, выполненные профессиональными дизайнерами. 

Совокупность перечисленных языков и библиотек позволяет сосредоточится на алгоритмах и правильной работе коде, а не тратить время на изучение огромного количества документации к интерфейсу, как в многих решениях для компилируемых языков программирования.
\subsubsection{Особенности реализации алгоритмов}
Граф представлен сеткой (англ. Grid) — массив клеток, который условно можно назвать картой. Каждая клетка является вершиной, имеющей 4 или 8 соседних вершин в зависимости от разрешённости диагонального движения. Координаты клетки определяются её положением в двухмерном массиве карты. Вес ребра равен дистанции между координатами [].

Рассмотрим клетку как прототип объекта. Определены поля $f$, $h$, $g$, $parent$, $closed$, $visited$, $type$, $i$, $j$. Значения первых трёх полей понятно из описания алгоритмов, приведённого выше. Поле $closed$ позволяет алгоритму понять окончательно обработана клетка или нет. Поле $visited$ необходимо для избежания повторного добавления клетки в открытый список (эквивалент множества $Q$). $i,j$ — координаты. $type$ — тип клетки, он может принимать 6 разных значений, но для алгоритма важно только является клетка проходимой или нет. Методы у клетки тривиальны и однотипны и завязаны на проверку клетки на тип, возвращают логический тип. 

Поиск пути реализован в объекте, состоящем из методов.
\subsubsection{Пример работы алгоритмов}
\subsubsection{Реализация интерфейса}